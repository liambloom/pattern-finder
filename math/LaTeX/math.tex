\documentclass[12pt, letterpaper]{article}
\usepackage{amsmath}
%\usepackage[utf8]{inputenc} this enables utf-8, and is currently commented out
\usepackage{cprotect}
\usepackage{hyperref}

\title{How this math works}
\author{Liam Bloom}
\date{\today}

\begin{document}

\tableofcontents

\section*{Introduction}

This library takes, as input, a series of $n$ points 
\((x_0, y_0), (x_1, y_1)\cdots \\(x_{n-2}, y_{n-2}), 
(x_{n-1}, y_{n-1})\), and returns {\textit a} \verb|Function| that 
produces these points. Note that this will not be the only such
\verb|Function|, only the one that my code feels makes the most sense. 
Here are, in order, the types of functions that may be returned, and
how they are calculated.

\section{Polynomial}

{\textbf Form:} \(f(x)=ax^{n-1}+bx^{n-2}\cdots cx^1+d\) \vspace{2ex}

A polynomial of the $n$-th degree, the standard form of which is shown
above, can be generated from a series of $\geq n$ points using linear
algebra. We begin by creating a system of equations:

\[y_0=ax_0^{n-1}+bx_0^{n-2}\cdots cx_0^1+d\]
\[y_1=ax_1^{n-1}+bx_1^{n-2}\cdots cx_1^1+d\]
\[\vdots\]
\[y_{n-2}=ax_{n-2}^{n-1}+bx_{n-2}^{n-2}\cdots cx_{n-2}^1+d\]
\[y_{n-1}=ax_{n-1}^{n-1}+bx_{n-1}^{n-2}\cdots cx_{n-1}^1+d\]

This can be rewritten as a linear algebra equation:

\[
    \begin{bmatrix} 
    x_0^{n-1} & x_0^{n-2} & \cdots & x_0^1 & 1 \\
    x_1^{n-1} & x_1^{n-2} & \cdots & x_1^1 & 1 \\
    \vdots & \vdots & \ddots & \vdots & \vdots \\
    x_{n-2}^{n-1} & x_{n-2}^{n-2} & \cdots & x_{n-2}^1 & 1 \\
    x_{n-1}^{n-1} & x_{n-1}^{n-2} & \cdots & x_{n-1}^1 & 1
    \end{bmatrix}
    \begin{bmatrix}
    a \\
    b \\
    \vdots \\
    c \\
    d
    \end{bmatrix}
    =
    \begin{bmatrix}
    y_0 \\
    y_1 \\
    \vdots \\
    y_{n-2} \\
    y_{n-1}
    \end{bmatrix}
\]

and solved by multiplying both sides by the inverse:

\[
    \begin{bmatrix}
    a \\
    b \\
    \vdots \\
    c \\
    d
    \end{bmatrix}
    =
    \begin{bmatrix} 
    x_0^{n-1} & x_0^{n-2} & \cdots & x_0^1 & 1 \\
    x_1^{n-1} & x_1^{n-2} & \cdots & x_1^1 & 1 \\
    \vdots & \vdots & \ddots & \vdots & \vdots \\
    x_{n-2}^{n-1} & x_{n-2}^{n-2} & \cdots & x_{n-2}^1 & 1 \\
    x_{n-1}^{n-1} & x_{n-1}^{n-2} & \cdots & x_{n-1}^1 & 1
    \end{bmatrix}^{-1}
    \begin{bmatrix}
    y_0 \\
    y_1 \\
    \vdots \\
    y_{n-2} \\
    y_{n-1}
    \end{bmatrix}
\]

We then plug \(a, b \cdots c, d\) into the standard form to generate 
our function.

\subsection{Old Polynomial Method}

The above linear algebra method was not used for determining polynomial
functions prior to commit \cprotect{\href{
https://github.com/liambloom/pattern-finder/commit/1f190e5aca4f2f4dda9d342b1e721a6458b44415
}}{\verb|1f190e5|}. Instead, the following method was used:

\section{Exponential}
{\textbf Form:} \(f(x)=ab^x+c\) \vspace{2ex}

An exponential equation, as shown above, can be generated from 
$\geq 3$ points

% TODO

\section{The Future}

This section lists what types of functions may or may not be added
to this library in the future, and if not, why.

\subsection{Likely to be added}
\begin{itemize}
    \item Rational
    \item Fibonacci-like
\end{itemize}


\subsection{May be added}
\begin{itemize}
    \item Trig
    \item Absolute Value
\end{itemize}

\subsection{Unlikely to be added}
\begin{itemize}
    \item Radical functions would most likely require storage of 
        irrational numbers, which is impossible
    \item Logarithmic functions for the same reason as radicals
    \item Conic Sections because they cannot be represented as functions
\end{itemize}

\end{document}